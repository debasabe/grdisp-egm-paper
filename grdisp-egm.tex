\documentclass[review,hidelinks,onefignum,onetabnum]{siamart220329}
%\input{ex_shared} %Can fin this file

\usepackage{amssymb}
\usepackage{amsmath}
\usepackage{graphicx}
\newcommand{\tw}{\textwidth}
\newcommand{\mitbf}[1]{\boldsymbol{\mathit{#1}}}
\newcommand{\dgpenalty}{R}
\newcommand{\dgsform}{S}
\graphicspath{{./FIG/},{../FIG/},{../Fractures/FIG/}}

\headers{Grid dispersion of EGM}{J.D.~{De Basabe}, J.~Vamaraju and M.K.~Sen}

\title{Grid-Dispersion Analysis of the Enriched-Galerkin Method for Elastic Wave Propagation}
\author{Jonas D.~De Basabe\thanks{CICESE, Ensenada, Mexico 
  (\email{jonas@cicese.mx}).}
\and Mrinal K.~Sen\thanks{The University of Texas at Austin, USA, (\email{mrinal@utexas.edu})}
\and Raul U.~Silva-Avalos\thanks{Universidad Autonoma de Zacatecas, Zacatecas, Mexico, (\email{rsilva@uaz.edu.mx})}
}

\date{\today}

\begin{document}

\maketitle

\begin{abstract}
The enriched Galerkin method (EGM) combines the advantages of spectral element and discontinuous Galerkin methods. 
We present a grid-dispersion analysis of this method and show that, using enriched basis of low polynomial order, the results are polluted with grid-dispersion error. This error can be minimized using high-order enrichment.
We also show that the stability conditions of EGM are significantly higher than those of the discontinuous Galerkin method.
Our analysis gives insight on the accuracy of EGM as a function of the order of the enriched basis functions.
\end{abstract}

\begin{keywords}
Enriched Galerkin method, grid dispersion, numerical stability.
MSC: 65L60, 37N30, 65M12, 35L10, 86A15, 74B05, 74J05, 74E10, 74A45.
\end{keywords}


\section{Introduction}
Initial paragraph about Finite Element Method...

In recent years, several FEM have been developed for effectively simulation of wave propagation on Earth, for acoustic and elastic medium (CITE SOME WORKS). However, FEM may present errors when complex domains are employed, e.g. crack and fracture modelling. More recently, the Discontinuous Galerkin Method (DGM) has been studied for modeling fracture media 

Paragraph about enrichment (constant, linear, polynomials)

Originated as eXtended FEM (eXtended FEM (XFEM), was proposed in 1999 by Ted Belytschko \cite{Belytschko2001}
originalmente planteado por \cite{Sun2009}

Paragraph about this work method

\section{The Elastic Wave Equation}
We consider an elastic, anisotropic and heterogeneous medium for the unbounded domain $\Omega\in\mathbb{R}^2$. The elastic wave equation is given by
%
\begin{eqnarray}\label{eq:elast_wv_eq}
\rho\partial_{tt}u_i=\partial_j\sigma_{ij}+f_i\ \mathrm{in}\ \Omega\times(0,\tau],
\end{eqnarray}
%
where $u_i=u_i(\mitbf x,t)$ is the displacement vector, $f_i=f_i(\mitbf x,t)$ is the body force, $\rho=\rho(\mitbf x)$ is the density and $\sigma_{ij}=\sigma_{ij}(\mitbf x,t)$ is the stress tensor. Following the generalized Hooke's law, the stress is linearly related to the strain as
%
\begin{eqnarray}\label{eq:stress_strain}
\sigma_{ij}=C_{ijkl}\epsilon_{kl},
\end{eqnarray}
%
where $C_{ijkl}=C_{ijkl}(\mitbf x)$ is the 4th order stiffness tensor with some symmetries and $\epsilon_{kl}=\epsilon_{kl}(\mitbf x,t)$ is the strain tensor defined as $\epsilon_{kl}=\left(u_{i,j}+u_{j,i}\right)/2$. Here the comma notation indicates differentiation.

\section{The Enriched-Galerkin Method}

The enriched Galerkin method (EGM) is a finite-element method that combines continuous and discontinuous basis functions. Its application to elastic wave propagation and \textit{a priori} error analysis is found in \cite{Vamaraju2018}. We include here the formulation of the method for completeness.

EGM is based on the following weak formulation of the elastic wave equation:
Find $\mitbf u\in \mitbf{X}^D$ such that for all $\mitbf v\in \mitbf{X}^D$
\begin{eqnarray}
\sum_{E\in \Omega_h} \Big( \left( \rho \partial_{tt} \mitbf u, \mitbf v \right )_E
	+{\mitbf B_E(\mitbf u, \mitbf v)} \Big)
	+ \sum_{\gamma\in\Gamma_h}{\mitbf J^c_\gamma(\mitbf u, \mitbf v; \dgsform, \dgpenalty)} =
	\sum_{E\in \Omega_h} ( \mitbf f, \mitbf v)_E
	\label{eq:elast_dg_weak}
\end{eqnarray}
where
$X^D = \left\{ \varphi \mid \varphi\in\mitbf H^1(E) \; \forall\; E\in \Omega_h,\;\varphi=0 \textrm{ on }\Gamma_D \right\}$,
\begin{eqnarray}
\mitbf B_E(\mitbf u, \mitbf v) &=&
\int_E C_{ijkl} u_{j,i} v_{l,k} \;d\Omega,
\\
{\mitbf J^c_\gamma(\mitbf u, \mitbf v; \dgsform, \dgpenalty)} &=&
- \int_{\gamma} \{\tau_i(\mitbf u)\}[v_i] \;d\gamma
+ \dgsform \int_{\gamma} \{\tau_i(\mitbf v)\}[u_i] \;d\gamma
\label{eq:jfunc}
\\ \nonumber
&+& \dgpenalty\int_{\gamma}\{C_{1111}\}[u_i][v_i]\;d\gamma,
\end{eqnarray}
$\tau_i(\mitbf u)=C_{ijkl} u_{l,k} n_j$ is the traction vector,
$C_{ijkl} = C_{ijkl}(\mitbf x)$ is the elastic tensor,
$\rho=\rho(\mitbf{x})$ is the density,
$\Omega_h$ is the set of elements and
$\Gamma_h$ is the set of all faces between elements.
The parameter $\dgpenalty$ is the penalty, and
$\dgsform$ is a parameter that takes the values
$\dgsform=0$ for the incomplete formulation,
$\dgsform=-1$ for the symmetric formulation and
$\dgsform=1$ for the non-symmetric formulation.
The penalty is given by 
$$
R= \sigma \frac{(\kappa + 1)(\kappa + 2)}{|\gamma|},
$$
where $|\gamma|$ is the length of the face, 
$\kappa$ is the polynomial order of the basis functions and 
$\sigma$ is an arbitrary constant parameter. 
In practice, $\sigma$ needs to be large enough to impose continuity across the elements. We will show in the stability analysis that, if $\sigma$ is too large, the stability conditions are more restrictive.



\section{Grid-Dispersion Analysis}

El problema semidiscreto
\begin{eqnarray*}
M_{ij}\partial_{tt} U^x_j &=& K^1_{ij} U^x_j +  K^2_{ij} U^z_j
\\
M_{ij}\partial_{tt} U^z_j &=& K^3_{ij} U^x_j +  K^4_{ij} U^z_j
\end{eqnarray*}
donde $M_{ij}$ es la matriz de masa y $K^\nu_{ij}$ son las matrices de rigidez.

Sustituyendo la soluci\'on de onda plana y usando los supuestos
\begin{eqnarray*}
\omega^2 M_{ij} U^x_j &=& \widetilde K^1_{ij} U^x_j +  \widetilde K^2_{ij} U^z_j
\\
\omega^2 M_{ij} U^z_j &=& \widetilde K^3_{ij} U^x_j +  \widetilde K^4_{ij} U^z_j
\end{eqnarray*}
donde $\widetilde K^\nu_{ij}$ son las matrices din\'amicas de rigidez, dadas por
\begin{eqnarray*}
	\widetilde K^\nu_{ij} = \widehat K^\nu_{ij} 
  + e^{ -\mathrm i k_z h} L^{\nu,T}_{ij} + e^{ \mathrm i k_z h} L^{\nu,B}_{ij} 
  + e^{ -\mathrm i k_x h} L^{\nu,L}_{ij} + e^{ \mathrm i k_x h} L^{\nu,R}_{ij},
\end{eqnarray*}
$L^{\nu,T}_{ij}$ son las matrices que se obtienen de las fronteras del elemento y $\widehat K^\nu_{ij}$ son las matrices din\'amicas de rigidez que corresponden a las bases continuas.

La dispersi\'on num\'erica est\'a relacionada a los eigenvalores de la siguiente ecuaci\'on
$$
\Lambda \mitbf M \mitbf U = \mitbf K \mitbf U 
$$
donde $\mitbf M$ es una matriz de masa de $2\eta \times 2\eta$, 
$\eta$ son los grados de libertad representativos del elemento de referencia, y
$\mitbf K$ contiene las matrices din\'amicas de rigidez.


\begin{figure}
\begin{picture}(150,150)
	\multiput(0,0)(50,0){4}{\line(0,1){150}}
	\multiput(0,0)(0,50){4}{\line(1,0){150}}
	%\multiput(50,0)(0,50){2}{\line(1,0){150}}
	%\multiput(0,50)(50,0){2}{\line(0,1){150}}
	\put(70,70){$E_0$}

	\put(120,70){$E_R$}
	\put( 20,70){$E_L$}
	\put(70, 20){$E_B$}
	\put(70,120){$E_T$}

	\put( 20,120){$E_{TL}$}
	\put(120,120){$E_{TR}$}
	\put( 20, 20){$E_{BL}$}
	\put(120, 20){$E_{BR}$}

	\put(100,70){$\gamma_{_R}$}
	\put( 40,70){$\gamma_{_L}$}
	\put(70, 42){$\gamma_{_B}$}
	\put(70,105){$\gamma_{_T}$}
\end{picture}
 
\caption{The reference element.}
\end{figure} 

\begin{figure}
\includegraphics[width=0.49\tw]{grdisp_egm1+1_gll_r=10.pdf}
\includegraphics[width=0.49\tw]{grdisp_sem1_r=10.pdf}
\caption{Curvas de dispersi\'on de EGM orden 1+1 y SEM orden 1, $r=10$}
\end{figure} 

\begin{figure}
\includegraphics[width=0.49\tw]{grdisp_egm2+1_gll_r=10.pdf}
\includegraphics[width=0.49\tw]{grdisp_egm2+2_gll_r=10.pdf}
\caption{Curvas de dispersi\'on de EGM orden 2+1 y 2+2, $r=10$}
\end{figure} 

\begin{figure}
\includegraphics[width=0.49\tw]{grdisp_sem2_r=10.pdf}
\includegraphics[width=0.49\tw]{grdisp_sipg2_r=10.pdf}
\caption{Curvas de dispersi\'on de SEM y DGM orden 2, $r=10$}
\end{figure} 

\begin{figure}
\includegraphics[width=0.49\tw]{grdisp_egm4+1_r=10.pdf}
\includegraphics[width=0.49\tw]{grdisp_egm4+2_r=10.pdf}
\caption{Curvas de dispersi\'on de EGM orden 4+1 y 4+2, $r=10$}
\end{figure} 

\begin{figure}
\caption{Curvas de dispersi\'on de EGM orden 4+3 y 4+4, $r=10$}
\includegraphics[width=0.49\tw]{grdisp_egm4+3_r=10.pdf}
\includegraphics[width=0.49\tw]{grdisp_egm4+4_r=10.pdf}
\end{figure} 

\begin{figure}
\caption{Curvas de dispersi\'on de SEM 4, $r=10$}
\includegraphics[width=0.49\tw]{grdisp_sem4_r=10.pdf}
\includegraphics[width=0.49\tw]{grdisp_sipg4_r=10.pdf}
\end{figure} 

\begin{figure}
\caption{Anisotrop\'ia num\'erica de los m\'etodos de orden 2. La dispersi\'on est\'a calculada a 10 longitudes de onda para prop\'ositos de visualizaci\'on y $\delta=0.2$}
\includegraphics[width=0.82\tw]{grdisp-aniso_o=2_wl=10}
\end{figure} 

\begin{figure}
\caption{Anisotrop\'ia num\'erica de los m\'etodos de orden 3. La dispersi\'on est\'a calculada a 100 longitudes de onda para prop\'ositos de visualizaci\'on y $\delta=0.2$}
\includegraphics[width=0.82\tw]{grdisp-aniso_o=3_wl=100}
\end{figure} 

\begin{figure}
\caption{Anisotrop\'ia num\'erica de los m\'etodos de orden 4.
La dispersi\'on est\'a calculada a 1000 longitudes de onda para prop\'ositos de visualizaci\'on y $\delta=0.2$}
\includegraphics[width=0.82\tw]{grdisp-aniso_o=4_wl=1000}
\end{figure} 

\section{Stability Conditions}

La condici\'on de estabilidad est\'a dada por
\begin{eqnarray*}
	q:=\frac{\alpha\Delta t}{h} \le C =
	\min_{1\le j\le 2\eta} 
	\min_{0\le\theta\le 2\pi}\; 2\Lambda_j(\theta)^{-1/2}
\end{eqnarray*}
donde $\Lambda_j$ son los eigenvalores, $\theta$ es el \'angulo de la onda plana con respecto a la malla y $h$ es el tama\~no de las caras de los elementos.

Condiciones de estabilidad (CFL), considerando el m\'etodo de diferencias finitas para avanzar en el tiempo.

\begin{table}
 
\caption{CFL Stability conditions of SEM, DGM (SIPG) and EGM.}

%\setlength{\extrarowheight}{1mm}
\begin{tabular}{|c | c  c | c  c  c  c |}
\hline
& & & \multicolumn{4}{c|}{EGM} 
\\
Orden & SEM & SIPG & N+1 & N+2 & N+3 & N+4 
\\
\hline
1 &	0.8821 &	0.2911 & 0.4117 & & &
\\
2 &	0.3330 &	0.1216 & 0.3022 &	0.1719 & &
\\
3 &	0.1897 &	0.0688 & 0.2339 &	0.1315 &	0.0973 &
\\
4 &	0.1204 &	0.0439 & 0.1703 &	0.1067 &	0.0790 &	0.0621
\\
5 &	0.0823 &	0.0308 & 0.1164 &	0.0899 &	0.0660 &	0.0514
\\
6 &	0.0595 &	0.0226 & 0.0841 &	0.0777 &	0.0568 &	0.0439
\\
8 &	0.0351 &	0.0138 & 0.0496 &	0.0496 &	0.0444 &	0.0342
\\
\hline
\end{tabular} 
\end{table} 
 
Estas condiciones son {necesarias pero no suficientes.}

\section{Numerical Simulations}

\begin{figure}
\includegraphics[width=0.45\tw]{exp1-eg8g1-ux-nt=967}
\includegraphics[width=0.45\tw]{exp1-eg8g1-uz-nt=967}
\caption{Campos de onda -- EGM orden 8+1, t=0.4 s \textbullet{} 967 pasos en tiempo \textbullet{} 170,801 nodos}
\end{figure} 

\begin{figure}
\includegraphics[width=0.45\tw]{exp1-eg8g2-ux-nt=967}
\includegraphics[width=0.45\tw]{exp1-eg8g2-uz-nt=967}
\caption{Campos de onda -- EGM orden 8+2, t=0.4 s \textbullet{} 967 pasos en tiempo \textbullet{} 183,301 nodos}
\end{figure} 

\begin{figure}
\includegraphics[width=0.45\tw]{exp1-eg8g4-ux-nt=967}
\includegraphics[width=0.45\tw]{exp1-eg8g4-uz-nt=967}
\caption{Campos de onda -- EGM orden 8+4, t=0.4 s \textbullet{} 967 pasos en tiempo \textbullet{} 223,301 nodos}
\end{figure} 

\section{Discusison}

\section{Conclusions}

\begin{itemize}
\item
Se plante\'o el m\'etodo de Galerkin enriquecido con bases enriquecida de orden elevado para medios el\'asticos y fracturados

\vspace{\stretch{1}}

\item
Para propagaci\'on de ondas, las bases de bajo orden polinomial introducen error de dispersi\'on num\'erica. Las bases continuas y las enriquecidas necesitan ser de orden superior

\vspace{\stretch{1}}

\item
Las condiciones de estabilidad de EGM son mayores que las de DGM. Son similares a las de SEM y en algunos casos superiores

\vspace{\stretch{1}}

\item
EGM puede incorporar discontinuidades relacionadas a las fracturas sin requerir tantos grados de libertad ni tama\~nos de paso peque\~nos como con DGM

\end{itemize}

\section*{Acknowledgments}
MFW

\bibliographystyle{siamplain}
\bibliography{mybib}

\end{document}
